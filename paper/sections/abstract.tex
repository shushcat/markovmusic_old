 \doublespace

 \section{Abstract} We hypothesize that treatment of individual measures of music\footnote{i.e. four beats in 4/4 time}, as the basic units of analysis across a large corpora of compositions, will allow generation of uniquely identifying probabilistic grammars. These grammars are composed of Markov-chains, which are generated by incrementing through each measure of a selected piece while tallying both that measure’s features, and those of the measures which immediately precede and succeed it. Initially, the radius, or neighborhood, is set at $r = 1$ so that one prior and following measure is evaluated for each increment. The neighborhood is then increased, until the hypothesis can be confirmed or rejected. Confirmation or rejection of the hypothesis occurs when a given radius either proves sufficient for intuitively satisfactory classification of analyzed pieces, or the radius is increased to such an extent that distinctions between pieces cease to be fine-grained enough to allow for comparison of pieces in our chosen corpora. Successful research in this area would entail quantitative characterization and comparison of musics both within, and across cultures.

