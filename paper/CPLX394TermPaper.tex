\documentclass{article}
\usepackage[utf8]{inputenc}     % Defines character encoding of input txt.
\usepackage[T1]{fontenc}            % Defines font encoding to use.
\usepackage{kpfonts}            % Beautiful fonts.
\usepackage{geometry}           % See geometry.pdf to learn the layout options. 
\geometry{letterpaper}          % ... or a4paper or a5paper or ... 
\usepackage{setspace}           % For controlling line spacing in the document.
\usepackage{amsmath}            % Extra math environments. Possibly unneeded.
\usepackage{amssymb}            % Some mathy symbols.
 \usepackage{titling}               % Allows fancyhdr to reference \thetitle. 
 \usepackage{fancyhdr}          % Lets me put the page count in the footer.
 \headheight = 14.49998pt           % Sets header to appropriate height.
 \pagestyle{fancy}              % Sets style for non-\maketitle pages.
 \lhead[]{} \chead[]{Working Title} \rhead[]{Brickley and Olsson}
 \lfoot[]{} \cfoot[]{\thepage\ of \pageref{LastPage}} \rfoot[]{}
 \usepackage{lastpage}          % Allows reference of last page number.
 \usepackage{hyperref}          % Allows links in the document.
 \usepackage[numbers]{natbib}               % Makes citations nonbad.
 %
 %
 \title{A Markov Model for Analysis of Musical Genre and Style}
\author{John O. Brickley and Wolfgang Q. Olsson}
\date{24 March 2013}

 \begin{document}
 \maketitle
 \doublespace
 %

 \section{Abstract} We hypothesize that treatment of individual measures of music\footnote{i.e. four beats in 4/4 time}, as the basic units of analysis across a large corpora of compositions, will allow generation of uniquely identifying probabilistic grammars. These grammars are composed of Markov-chains, which are generated by incrementing through each measure of a selected piece while tallying both that measure’s features, and those of the measures which immediately precede and succeed it. Initially, the radius, or neighborhood, is set at $r = 1$ so that one prior and following measure is evaluated for each increment. The neighborhood is then increased, until the hypothesis can be confirmed or rejected. Confirmation or rejection of the hypothesis occurs when a given radius either proves sufficient for intuitively satisfactory classification of analyzed pieces, or the radius is increased to such an extent that distinctions between pieces cease to be fine-grained enough to allow for comparison of pieces in our chosen corpora. Successful research in this area would entail quantitative characterization and comparison of musics both within, and across cultures.


 %
 \singlespace
 \section{Introduction and Background}
 \begin{itemize}
     \item Introduce relevant topics.
     \item Justify question.
     \item Literature Review
     \item other approaches, methods
     \item your approach and justification
     \item details of the methods you are using
     \item any additional information needed before the rest of the article
     \item basic definitions
     \item topical introductions
 \end{itemize}
 \subsection{To Include}
 \begin{itemize}
     \item possible approaches,
     \item approaches employed by others in answering similar questions,
     \item in particular, in which respects other research has already succeeded or failed in settling out hypothesis,
     \item \citep{Russell2009AIMA} Chapter 15, Probabilistic Reasoning Over Time chapter in Artificial Intelligence.
     \item \citep{Poliner2005CAMT} A Classification Approach to Melody Transcription. History.
     \item for later \citep{Metze2013FALFED} Fusion of Acoustic and Linguistic Speech Features for Emotion Detection

 \end{itemize}

\section{Methodology}

\section{Experiments}

\section{Results}

\section{Analysis and Evaluation}
\section{Discussion}

 \end{document}  
